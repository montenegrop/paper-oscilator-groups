\documentclass[12pt]{amsart}
\usepackage{amsopn,amsmath,amssymb,amsthm,eucal,url,amscd,amsgen}
\usepackage{enumerate}
\usepackage[pagebackref]{hyperref}
\usepackage[arrow]{xy} % Function diagrams
\usepackage{setspace} % Space after section
\usepackage{multirow} % For tables
\usepackage{xfrac} % Diagonal fractions sfrac
\usepackage{nccmath} % Center equations
\usepackage{enumitem} % Spacing
\usepackage{xcolor} % To do stuff

\DeclareMathOperator{\sign}{sign} % Sign operator

\newcommand\myworries[1]{\textcolor{red}{#1}}

\makeatletter
\newcommand{\mylabel}[2]{#2\def\@currentlabel{#2}\label{#1}}
\makeatother

\textwidth 14cm 
\textheight 20cm
\oddsidemargin .4in
\evensidemargin .4in

\newcommand{\nc}{\newcommand}
\renewcommand{\aa}{\mathfrak{a}} \newcommand\aff{{\mathfrak{aff}}}
\nc{\bb}{\mathfrak{b}} \nc{\cc}{\mathfrak{c}} \nc{\dd}{\mathfrak{d}}
\newcommand\ee{{\mathfrak e}} \nc{\ggo}{\mathfrak{g}}
\nc{\hh}{\mathfrak{h}} \nc{\ii}{\mathfrak{i}}
\nc{\jj}{\mathfrak{j}} \nc{\kk}{\mathfrak{k}}
\nc{\mm}{\mathfrak{m}} \nc{\nn}{\mathfrak{n}}
\nc{\pp}{\mathfrak{p}} \newcommand\qq{{\mathfrak q}}
\nc{\rr}{\mathfrak{r}} \nc{\sg}{\mathfrak{s}}
\nc{\sso}{\mathfrak{so}} \nc{\spg}{\mathfrak{sp}}
\nc{\ssu}{\mathfrak{su}} \nc{\ssl}{\mathfrak{sl}}
\nc{\tog}{\mathfrak{t}} \nc{\uu}{\mathfrak{u}}
\nc{\vv}{\mathfrak{v}} \nc{\ww}{\mathfrak{w}}
\nc{\zz}{\mathfrak{z}}

\newcommand\norm[1]{\lVert#1\rVert}
\newcommand\normx[1]{\Vert#1\Vert}

\newcommand{\ggam}{G/\Gamma}
\nc{\CC}{{\mathbb C}}
\nc{\DD}{{\mathbb D}}
\nc{\FF}{{\mathbb F}}
\nc{\GG}{{\mathbb G}}
\nc{\HH}{{\mathbb H}}
\nc{\II}{{\mathbb I}}
\nc{\JJ}{{\mathbb J}}
\nc{\KK}{{\mathbb K}}
\nc{\NN}{{\mathbb N}}
\newcommand\QQ{\mathbb Q}
\nc{\RR}{{\mathbb R}}
\nc{\ZZ}{{\mathbb Z}}

\newcommand{\Heis}{\mathrm{H}}

\nc{\ggob}{\overline{\mathfrak{g}}}
\nc{\glg}{\mathfrak{gl}}

\nc{\pca}{\mathcal{P}} \nc{\nca}{\mathcal{N}}

\nc{\vp}{\varphi} \nc{\ddt}{\frac{{\rm d}}{{\rm d}t}}
\nc{\la}{\langle} \nc{\ra}{\rangle}
\nc{\brg}{[\,,\,]_{\ggo}}
\nc{\brv}{[\,,\,]_{\vv}}

\nc{\SO}{{\sf SO}} \nc{\Spe}{{\sf Sp}} \nc{\Sl}{{\sf Sl}}
\nc{\SU}{{\sf SU}} \nc{\Or}{{\sf O}} \nc{\U}{{\sf U}}
\nc{\Gl}{{\sf Gl}} \nc{\Se}{{\sf S}} \nc{\Cl}{{\sf Cl}}
\nc{\Spin}{{\sf Spin}} \nc{\Pin}{{\sf Pin}}

% Operators
\nc{\sldr}{\operatorname{SL(2,\R)}}
\nc{\sldrt}{\operatorname{\widetilde{SL}(2,\R)}}
\nc{\Gamt}{\operatorname{\widetilde{\Gamma}}}
\nc{\alpt}{\operatorname{\widetilde{\alpha}}}
\nc{\gsldr}{\operatorname{\mathfrak{sl}(2,\R)}}
\nc{\gldr}{\operatorname{GL(2,\R)}}
\nc{\sldz}{\operatorname{SL(2,\Z)}}
\nc{\B}{\operatorname{B}}
\nc{\oscn}{\operatorname{Osc_n(\lambda_1,...,\lambda_n)}}

\nc{\ad}{\operatorname{ad}} \nc{\Ad}{\operatorname{Ad}}
\nc{\coad}{\operatorname{coad}}
\nc{\rank}{\operatorname{rank}} \nc{\Irr}{\operatorname{Irr}}
\nc{\End}{\operatorname{End}} \nc{\Aut}{\operatorname{Aut}}
\nc{\Inn}{\operatorname{Inn}} \nc{\Der}{\operatorname{Der}}
\nc{\Ker}{\operatorname{Ker}} \nc{\Iso}{\operatorname{Iso}}
\nc{\Le}{\operatorname{L}} \nc{\Fe}{\operatorname{F}}
\nc{\tr}{\operatorname{tr}}
\nc{\dif}{\operatorname{d}} \nc{\sen}{\operatorname{sen}}
\nc{\modu}{\operatorname{mod}} \nc{\Ric}{\operatorname{R}}
\nc{\Sym}{\operatorname{Sym}} \nc{\sca}{\operatorname{sc}}
\nc{\scalar}{{\sf s}} \nc{\grad}{\operatorname{grad}}
\nc{\ricci}{\operatorname{r}} \nc{\riccin}{\operatorname{Ric}}
\nc{\Lie}{\operatorname{L}} \nc{\ct}{\operatorname{T}}

\newenvironment{proof1}{\noindent {\textit{Proof of Theorem \ref{connected}:}}}{\hfill $\blacksquare$\bigskip}

\newcommand{\deax}{\partial_x}
\newcommand{\deay}{\partial_y}
\newcommand{\deaz}{\partial_z}
\newcommand{\deat}{\partial_t}

% Commands
\newcommand{\cen}{\mathfrak{z}(\mathfrak{g})}
\newcommand{\rad}{\mathfrak{r}}
\newcommand{\sem}{\mathfrak{s}}
\newcommand{\meti}{\left\langle}
\newcommand{\metd}{\right\rangle}
\newcommand{\lela}{\left \langle}
\newcommand{\rira}{\right \rangle}
\newcommand{\bil}{\lela\,,\,\rira}
\newcommand{\tf}{\tilde{f}}
\nc{\mr}{{\mathfrak r}}
\nc{\ms}{{\mathfrak s}}
\nc{\mv}{{\mathfrak v}}
\nc{\lra}{\longrightarrow}
\nc{\R}{{\mathbb R}}
\nc{\Q}{{\mathbb Q}}
\nc{\Z}{{\mathbb Z}}
\newcommand{\mX}{\mathfrak X}
\newcommand{\mF}{\mathfrak F}
\newcommand{\mg}{\mathfrak n}
\newcommand{\mn}{\mathfrak n}
\newcommand{\mz}{\mathfrak z}

\newcommand{\mh}{\mathfrak h}
\newcommand{\ma}{\mathfrak a}
\newcommand{\mgg}{\mathfrak g}
\newcommand{\mt}{\mathfrak t}
\newcommand{\mb}{\mathfrak b}
\newcommand{\ts}{\mathfrak{ts}}
\newcommand{\bsh}{\backslash}

\nc{\hs}{{G/\Gamma}}

% Theorem styles
\theoremstyle{plain}
\newtheorem{thm}{Theorem}[section]
\newtheorem{prop}[thm]{Proposition}
\newtheorem{cor}[thm]{Corollary}
\newtheorem{lem}[thm]{Lemma}

\theoremstyle{definition}
\newtheorem{defn}[thm]{Definition}

\theoremstyle{remark}
\newtheorem{rem}{Remark}
\newtheorem*{rems}{Remarks}
\newtheorem{exa}[thm]{Example}
\newtheorem{exams}[thm]{Examples}
\newtheorem*{nota}{Note}
\newtheorem{obs}[thm]{Observations}

\newcommand{\ri}{{\rm (i)}}
\newcommand{\rii}{{\rm (ii)}}
\newcommand{\riii}{{\rm (iii)}}
\newcommand{\riv}{{\rm (iv)}}
\newcommand{\rv}{{\rm (v)}}
\newcommand{\script}{\scriptstyle}

\begin{document}


\title[Geodesics and Isometries on Compact Lorentzian Solvmanifolds]{Geodesics and Isometries on Compact Lorentzian Solvmanifolds}

\begin{abstract}
This work studies geodesics on Lorentzian homogeneous spaces of the form $M=G/\Lambda$, where $G$ is a solvable Lie group endowed with a bi-invariant Lorentzian metric and $\Lambda \subset G$ is a cocompact lattice. We provide conditions under which lightlike, timelike, or spacelike geodesics on the compact quotient spaces are closed. This study implicitly requires additional information about the lattices in each case. We establish conditions under which every lightlike geodesic on the quotient space is closed. More importantly, this situation depends on the lattice. Moreover, we provide examples in dimension four where not every lightlike geodesic is closed.
\end{abstract}

\author{Pablo Montenegro and Gabriela P. Ovando}

\let\today\relax
\thanks{{\it (2000) Mathematics Subject Classification}: 53C50, 53C22, 22F30, 57S25}
\thanks{{\it Key words and phrases}: Lorentzian geometry, geodesics, compact solvmanifolds.}
\thanks{Partially supported by ANPCyT, SCyT-UNR.}

\address{Departamento de Matemática, ECEN - FCEIA, Universidad Nacional de Rosario, Pellegrini 250, 2000 Rosario, Santa Fe, Argentina.}
\email{gabriela@fceia.unr.edu.ar}

\maketitle

\section{Introduction}
A Lorentzian manifold is a connected, smooth, finite-dimensional manifold $(M, \la \, , \ \ra)$ endowed with a Lorentzian metric, which is a second-order smooth tensor field on $M$ inducing, for every $p\in M$, a bilinear form of index 1 on the tangent space $T_pM$ (see \cite{ON}). The geodesics on $M$ are smooth curves $\gamma(t)$ that satisfy the differential equation
\begin{equation*}
\nabla_{\gamma'(t)}\gamma'(t)=0,
\end{equation*}
where $\nabla$ denotes the Levi-Civita connection for $\gamma$. The study of Lorentzian manifolds is of particular interest because models of space-time in general relativity are four-dimensional Lorentzian manifolds. In this context, the nature of a geodesic is determined by its initial conditions. Thus, a timelike geodesic, for which $\la \gamma'(t), \gamma'(t)\ra < 0$, represents the world line of a particle under the influence of a gravitational field. If $\la \gamma'(t), \gamma'(t)\ra = 0$, the geodesic is called lightlike or null, representing the world line of a light ray.

The study of closed geodesics is a classical topic in Riemannian geometry and has also been explored in Lorentzian geometry using different techniques. Galloway \cite{Ga} showed that any closed Lorentzian surface (compact and without boundary) contains at least one closed timelike or lightlike periodic geodesic. In \cite{Su}, Suhr proved that every closed Lorentzian surface contains two closed geodesics, one of which is definite, i.e., timelike or spacelike. Indeed, there are many open questions.

Lie groups are valuable tools for studying specific geometric behaviors. In particular, in \cite{BOV}, the authors show families of compact Lorentzian manifolds for which every lightlike geodesic is closed. Motivated by these results, we investigate the situation in higher dimensions and for various lattices in this paper. Specifically, we consider oscillator groups of dimension $2n+2$, equipped with a Lorentzian bi-invariant metric, and discrete subgroups such that the corresponding quotient space $M$ is compact. We refine the aforementioned result by demonstrating that the closedness of lightlike geodesics depends on the choice of lattice. Precisely, Theorem \ref{teoremaoscilador} provides a condition on the lattice that implies either every lightlike geodesic in the compact manifold $M$ is closed, or there is exactly one direction for which lightlike geodesics are closed, while in any other direction, they are not closed.

To complete the study, we determine the existence of closed and open timelike and spacelike geodesics. In Theorem \ref{othergeodesics}, we prove that every type of such geodesic exists and provide explicit examples for each case.

Finally, we examine the isometries of the compact quotients. It was proved in \cite{BG} that the identity component of the isometry group of a pseudo-Riemannian compact space coincides with $G$ whenever $G$ is a solvable Lie group acting by isometries. We base our study on the results obtained in \cite{Bou}, where the isometry groups of the oscillator Lie groups were computed when considered with a bi-invariant metric. By generalizing the results from \cite{BOV}, we compute the isometry groups for certain compact spaces. We observe that isometries fixing the identity element in oscillator groups strictly include the conjugation maps (see Theorem 4.6). However, inducing isometries to the quotient spaces requires conjugation by an element of the normalizer of the corresponding lattice (see Proposition 4.11). Additionally, any left-translation is naturally induced to the quotient. Computations of the normalizer of the lattices become much more intricate in higher dimensions. In the final section, we provide some examples of these computations.

	
	
\section{Lie groups with Lorentzian bi-invariant metrics}\label{section1}

In this section, we introduce general results about Lie groups with bi-invariant Lorentzian metrics.

Let \( G \) denote a (real) Lie group with Lie algebra \( \mathfrak{g} \). A \textit{bi-invariant} metric on \( G \) is a pseudo-Riemannian metric \( \langle \,,\, \rangle \) for which every left translation \( L_g \) and right translation \( R_g \) by elements of the group \( g \in G \) are isometries. This implies that the conjugation maps \( I_g: G \to G \), \( I_g(x) = gxg^{-1} \), are isometries. Hence, the differential of the Adjoint map is a linear isometry on \( \mathfrak{g} \), \( d(I_g)_e = \text{Ad}(g) \). The following equivalences hold (see Chapter 11 in \cite{ON}):

\begin{enumerate}\label{[(i)]}
    \item \( \langle \,,\, \rangle \) is bi-invariant;
    \item \( \langle \,,\, \rangle \) is Ad(\( G \))-invariant;
    \item \( \langle [X, Y], Z \rangle + \langle Y, [X, Z] \rangle = 0 \) for all \( X, Y, Z \in \mathfrak{g} \);
    \item The geodesics of \( G \) starting at the identity element \( e \) are the one-parameter subgroups of \( G \), that is:
    \begin{equation}\label{onepara}
        \alpha(t) = \exp(tX), \qquad \text{for } X \in \mathfrak{g},
    \end{equation}
    and the geodesic through \( g \in G \) with initial left-invariant vector \( X \) is given by the translation of the curve above, \( g\exp(tX) \).
\end{enumerate}

If the bi-invariant metric on a Lie group \( G \) of dimension \( n \) has signature \( (1, n-1) \), the metric is called a \textit{Lorentzian metric}.

A given vector field \( X \in TG \) is called:
\begin{itemize}
    \item \textit{spacelike} whenever \( \langle X, X \rangle > 0 \);
    \item \textit{timelike} whenever \( \langle X, X \rangle < 0 \);
    \item \textit{lightlike} or \textit{null} if \( \langle X, X \rangle = 0 \).
\end{itemize}

This classification extends to geodesics: a geodesic on \( G \) with initial condition \( X \), namely \( \gamma_X(t) \), is called \textit{spacelike}, \textit{timelike}, or \textit{lightlike} if \( X \) belongs to the respective class above.

Examples of Lie groups with bi-invariant Lorentzian metrics arise from the so-called \textit{oscillator groups}. Denoted by \( \text{Osc}_n(\lambda_1, \ldots, \lambda_n) \), an oscillator Lie group is the simply connected Lie group with real Lie algebra of dimension \( 2n + 2 \), namely \( \mathfrak{osc}_n(\lambda_1, \ldots, \lambda_n) \), with \( \lambda_i \in \mathbb{R}_{>0} \). This Lie algebra is spanned by the basis \( Z, \{X_i, Y_i\}_{i=1}^n, T \) satisfying the non-trivial Lie bracket relations
\[
[X_i, Y_i] = Z, \quad [T, X_i] = \lambda_i Y_i, \quad [T, Y_i] = -\lambda_i X_i.
\]
Denote by \( \langle\,,\, \rangle \) the ad-invariant metric on \( \mathfrak{osc}_n(\lambda_1, \ldots, \lambda_n) \) with the non-zero relations
\[
\lambda_i \langle X_i, X_i \rangle = \lambda_i \langle Y_i, Y_i \rangle = \langle Z, T \rangle = 1.
\]

The oscillator Lie groups have the differential structure of \( \mathbb{R} \times \mathbb{R}^{2n} \times \mathbb{R} \) with the following group product
\begin{equation*}
(z_1, v_1, t_1) \cdot (z_2, v_2, t_2) = \left(z_1 + z_2 + \frac{1}{2} v_1^{\tau} J R(t_1) v_2, v_1 + R(t_1) v_2, t_1 + t_2\right),
\end{equation*}
\[
\text{where } R(t_1) = e^{t_1 N_{\lambda}}, \quad N_\lambda = \left(\begin{matrix}
    J_{\lambda_1} &  & \mathbf{0} \\
    & \ddots & \\
    \mathbf{0} & & J_{\lambda_n}
\end{matrix}\right), \quad
J_{\lambda_i} = \left(\begin{matrix}
    0 & -\lambda_i \\
    \lambda_i & 0
\end{matrix}\right), \quad J = N_{(-1, \ldots, -1)}.
\]

for \( v_1, v_2 \in \mathbb{R}^{2n} \). By \( v^{\tau} \), we denote the transpose of \( v \). Take the corresponding left-invariant metric on the Lie group, which, in usual coordinates for \( i = 1, \ldots, n \): \( z, x_i, y_i, t \) in \( \mathbb{R}^{2n+2} \), can be written as
\begin{equation}\label{metricosc}
    g = dt \left(dz + \sum_{j=1}^{n} y_j dx_j + x_j dy_j\right) + \sum_{j=1}^{n} \frac{1}{\lambda_j} \left(dx_j^2 + dy_j^2\right).
\end{equation}

The Christoffel symbols corresponding to the metric above are:

\[
\Gamma^1_{2n+2 \,\, 2i} = -\frac{x_i \lambda_i}{4}, \quad \Gamma^1_{2n+2 \,\, 2i+1} = -\frac{y_i \lambda_i}{4}, \quad i = 1, \ldots, n,
\]

\[
\Gamma^{2i}_{2n+2 \,\, 2i} = \frac{\lambda_i}{2}, \quad \Gamma^{2i+1}_{2n+2 \,\, 2i} = -\frac{\lambda_i}{2}, \quad i = 1, \ldots, n,
\]

with the others being trivial and following symmetry relations.

The resulting equations for the geodesics can be written in the usual coordinates of \( \mathbb{R}^{2n} \) as:

\begin{equation}\label{geodcomp}
\begin{aligned}
    z''(s) & = \frac{t'(s)}{2} \sum_{k=1}^{n} \lambda_k \left(x_k'(s) x_k(s) + y_k'(s) y_k(s)\right), \\ \vspace{.2cm}
    x_i''(s) & = -\lambda_i y_i'(s) t'(s), \\ \vspace{.2cm}
    y_i''(s) & = \lambda_i x_i'(s) t'(s), \\ \vspace{.2cm}
    t''(s) & = 0,
\end{aligned}
\end{equation}

which follows from the general geodesic equation:

\[
\frac{d^2 \gamma^k}{dt^2} + \sum_{i,j} \Gamma^k_{ij}(\gamma) \frac{d \gamma^i}{dt} \frac{d \gamma^j}{dt} = 0 \quad \text{(see \cite{ON})}.
\]

In particular, the geodesics \( \gamma_X(s) = (z(s), (x_j(s), y_j(s)), t(s)) \), \( j = 1, \ldots, n \), starting at the identity element with initial condition \( X = d Z + \sum_j (b_j X_j + c_j Y_j) + a T \) are:

\begin{itemize}
    \item For \( a \neq 0 \):
    \begin{eqnarray}\label{geo_osc_1}
        z(s) & = & \left(d + \frac{1}{2 a} \sum_{k=1}^{n} \frac{b_k^2 + c_k^2}{\lambda_k}\right)s - \frac{1}{2 a^2} \left(\sum_{k=1}^{n} \frac{b_k^2 + c_k^2}{\lambda_k^2} \sin(\lambda_k a s)\right), \\
        x_j(s) & = & \frac{1}{a \lambda_j} \left(b_j \sin(\lambda_j a s) + c_j \cos(\lambda_j a s) - c_j\right), \\
        y_j(s) & = & \frac{1}{a \lambda_j} \left(-b_j \cos(\lambda_j a s) + c_j \sin(\lambda_j a s) + b_j\right), \\
        t(s) & = & a s,
    \end{eqnarray}

    \item While for \( a = 0 \), one has:
    \begin{equation}\label{geo2}
        (z, (x_j, y_j), t)(s) = (ds, (b_j s, c_j s), 0).
    \end{equation}
\end{itemize}

It is not hard to check that for the initial velocity \( X \in \mathfrak{osc}_n(\lambda_1, \ldots, \lambda_n) \) as above, the corresponding geodesic is:
\begin{itemize}
    \item \textit{lightlike} if \( 2ad + \sum_{k=1}^{n} \frac{b_k^2 + c_k^2}{\lambda_k} = 0 \),
    \item \textit{timelike} if \( 2ad + \sum_{k=1}^{n} \frac{b_k^2 + c_k^2}{\lambda_k} < 0 \),
    \item or \textit{spacelike} if \( 2ad + \sum_{k=1}^{n} \frac{b_k^2 + c_k^2}{\lambda_k} > 0 \).
\end{itemize}

Note that the oscillator Lie groups are also complete spaces.
\smallskip

\begin{rem}
    Medina and Revoy in \cite{Me,MeRe} proved that the Lie algebras \( \mathfrak{osc}_n(\lambda_1, \ldots, \lambda_n) \) (\( \lambda_i > 0 \)) and \( \mathfrak{sl}(2, \mathbb{R}) \) are the only indecomposable ones admitting a Lorentzian ad-invariant metric. Recall that a Lie algebra provided with a metric is called \textbf{indecomposable} if the restriction of the metric to any proper ideal is degenerate.
\end{rem}

\subsection{Quotient spaces}

Let \( G \) denote a Lie group, and let \( \Gamma \subset G \) be a discrete cocompact subgroup. The quotient space \( M = G/\Gamma \) consists of elements of the form \( g\Gamma \) with \( g \in G \). Since \( \Gamma \) is closed, there exists a unique manifold structure on \( M \) for which the canonical projection \( g \mapsto g\Gamma \) is a smooth submersion (see \cite{Hel}). Finally, the geometry of \( M \) is provided by requiring the projection, named \( \pi \), to be a local isometry. Whenever the Lie group \( G \) is provided with a Lorentzian metric, \( (G, \pi) \) is called a \textit{Lorentzian covering} of \( M \).

Assume \( G \) is equipped with a bi-invariant metric. It follows that the geodesics of \( M \) starting at \( o := \pi(e) \) are of the form \( \hat{\alpha} = \pi(\alpha(t)) \), where \( \alpha \) is a one-parameter subgroup of \( G \) (see \cite{ON}). In addition to this, \( G \) acts on \( M \) by the "translations on the left," which are isometries:
\begin{equation*}
    \tau_g : M \rightarrow M, \quad \text{given by} \quad \tau_g(h\Gamma) := gh\Gamma,
\end{equation*}
showing that \( M = G/\Gamma \) is a homogeneous space.

One can notice that:

\begin{enumerate}
    \item A geodesic of \( G/\Gamma \) starting at \( g\Gamma \) is the translation via \( \tau_g \) of some geodesic starting at \( o \). \label{punto1}
    \item Every geodesic in \( G/\Gamma \) is the projection via \( \pi \) of some geodesic in \( G \). \label{punto2}
    \item Lightlike, timelike, and spacelike geodesics of \( G \) project to lightlike, timelike, and spacelike geodesics of \( M \), respectively.
\end{enumerate}

Since \( \pi \circ L_g = \tau_g \pi \), one gets that \( \tau(g)\pi \circ \alpha = \pi \circ L_g \circ \alpha \) for a curve \( \alpha: (a,b) \to G \) starting at the identity element \( e \in G \).

\begin{enumerate}
    \setcounter{enumi}{3}
    \item A geodesic \( \alpha: (-\varepsilon, \varepsilon) \to G \), with \( \varepsilon > 0 \) and \( \alpha(0) = e \), giving rise to the curve \( \pi \circ \alpha \) in \( M \) is non-simple in \( M \) if and only if \( \alpha(t) \in \Gamma \) for some \( t > 0 \). \label{punto4}
\end{enumerate}

In particular, the projection of a non-simple geodesic in \( G \) is always a closed curve in \( M \).

A final result for non-simple geodesics comes from the following lemma which, when combined with item (\ref{punto4}), states that every non-simple geodesic in the quotient manifold is actually a periodic curve.

\begin{lem}[\cite{BOV}]
    Let \( G \) be a Lie group, let \( K < G \) be any closed Lie subgroup of \( G \) such that \( \pi: G \to G/K \) denotes the usual projection. Let \( \alpha: \mathbb{R} \to G \) denote a one-parameter subgroup of \( G \). If \( \pi \circ \alpha \) is non-simple in \( G/K \), then it is periodic.
\end{lem}

In this paper, \textit{closed} geodesics will be periodic ones.

	

	
\section{The solvmanifolds from the Oscillator groups}\label{sectionosc}

This section is concerned with the study of geodesics of Lorentzian compact spaces 
\[
M = \text{Osc}_n(\lambda_1, \ldots, \lambda_n) / \Gamma,
\]
where \( \Gamma \) is a cocompact lattice in \( \text{Osc}_n(\lambda_1, \ldots, \lambda_n) \). The following result shows a condition to construct such lattices.

\begin{lem}[\cite{MeRe}]\label{lema_medina}
    An oscillator group \( \text{Osc}_n(\lambda_1, \ldots, \lambda_n) \) admits a lattice if and only if the numbers \( \lambda_j \) generate an additive discrete subgroup of \( \mathbb{R} \).
\end{lem}

In the demonstration of the previous lemma, it is shown that for a lattice \( \Gamma \), the set \( \mathrm{T}(\Gamma) := \{ t \in \mathbb{R} : (z,u,t) \in \Gamma \} \) is an additive discrete subgroup of \( \mathbb{R} \). Let \( t_0 \) denote the positive generator of \( \mathrm{T}(\Gamma) \).

Notice that for \( (w,b,0) \in \Gamma \), the set of elements in the lattice
\[
\{ (z,u,t_0)^n \cdot (w,b,0) \cdot (z,u,t_0)^{-n} = (w,e^{n t_0 N_{\lambda}}b,0) : n \in \mathbb{N} \}
\]
is a finite set, since they are elements of a discrete cocompact lattice. Furthermore, there is a smallest positive integer \( K_0 \) such that \( e^{K_0 t_0 N_{\lambda}} = \text{Id} \). In particular, it follows that \( t_0 \) satisfies
\begin{equation}\label{oscilator-N}
    t_0 = \frac{2 \pi k_i}{K_0 \lambda_i},
\end{equation}
for some integers \( k_i \) with \( i = 1, \ldots, n \).

In \cite{MF}, Fischer introduced a family of Lie groups named \( \text{Osc}_n(\omega_r, B_r) \) defined by an element \( r = (r_1, \ldots, r_n) \in \mathbb{N}^n \) such that \( r_i \mid r_{i+1} \). Denote by \( \omega_r(u,v) := u^T N_{-r} v \) the symplectic form on \( \mathbb{R}^{2n} \) and by \( B_r \in GL(2n, \mathbb{R}) \) the linear transformation satisfying:

\begin{itemize}
    \item \( \omega_r(B_r \cdot, \cdot) \) is symmetric and negative definite.
    \item \( e^{B_r} \in SL(2n, \mathbb{Z}) \).
\end{itemize}

The group operation for \( \text{Osc}_n(\omega_r, B_r) \), based on the manifold \( \mathbb{R} \oplus \mathbb{R}^{2n} \oplus \mathbb{R} \), is given by
\begin{equation}
    (z_1, v_1, t_1) \cdot (z_2, v_2, t_2) = \left(z_1 + z_2 + \frac{1}{2} v_1^T N_{-r} e^{t_1 B_r} v_2, v_1 + e^{t_1 B_r} v_2, t_1 + t_2\right).
\end{equation}

Let \( L(\xi_0) \) be the subgroup generated by 
\[
\{ (1,0,0), (0,e_i,0), (0,\xi_0,1) \},
\]
where \( \xi_0 \) is an element in \( \mathbb{R}^{2n} \) such that the above subgroup is a lattice.

In particular, according to Example 3.1 of \cite{MF}, the element \( \xi_0 \) satisfies the following condition:
\begin{equation}\label{xi-condition}
    (\omega_r(\xi_0, e^{B_r}e_i), e^{B_r} e_i, 0) \in \,\, \langle (1,0,0), (0,e_i,0) \rangle.
\end{equation}

	
	
\section{The solvmanifolds from the Oscillator groups}\label{sectionosc}

We assert that the lattices \( L(\xi_0) \) of \( \text{Osc}_n(\omega_r, B_r) \) can be associated with lattices of \( \text{Osc}_n(\lambda_1, \ldots, \lambda_n) \). In fact, for every lattice \( \Gamma \) of \( \text{Osc}_n(\lambda_1, \ldots, \lambda_n) \), there exists a group \( \text{Osc}_n(\omega_r, B_r) \), \( \xi_0 \in \mathbb{R}^{2n} \), and an isomorphism \( \Phi: \text{Osc}_n(\lambda_1, \ldots, \lambda_n) \rightarrow \text{Osc}_n(\omega_r, B_r) \) such that \( \Phi(\Gamma) = L(\xi_0) \) (see Theorem 5 of \cite{MF}). The explicit definition of \( \Phi \) can be found in the proof of the mentioned theorem. Moreover, the following property of this isomorphism holds:

\begin{equation} \label{condition-exp}
    \Phi^{-1}(z,0,t) = (w z, 0, \widetilde{t_0} t) \quad \text{whenever } e^{t B_r} = \text{Id},
\end{equation}
where \( \widetilde{t_0} \) is either \( \frac{1}{t_0} \) or \( -\frac{1}{t_0} \).

Additionally, it is shown that \( B_r := \pm t_0 S N_{\lambda} S^{-1} \)\footnote{This follows by noticing that \( \text{Osc}_n(\lambda_1, \ldots, \lambda_n) = \text{Osc}_n(\omega_1, N_{\lambda}) \).}, for some invertible matrix \( S \).

\begin{lem}[\cite{MF}]\label{oscilador-elementos}
    Let \( \Gamma \) be any lattice of \( \text{Osc}_n(\lambda_1, \ldots, \lambda_n) \). Then:
    \begin{enumerate}
        \item There always exists \( w \neq 0 \in \mathbb{R} \) such that \( (w, 0, 0) \in \Gamma \).
        \item If \( K_0 = 1 \), implicitly defined in Equation \eqref{oscilator-N}, then there exists an element in \( \Gamma \) of the form \( \gamma = (z, 0, t) \), where \( z \) and \( t \) are non-zero.
    \end{enumerate}
\end{lem}

\begin{proof}

    Since \( \Gamma = \Phi^{-1}(L(\xi_0)) \), then \( \Phi^{-1}(1,0,0) = (w,0,0) \in \Gamma \), according to \eqref{condition-exp}, with \( w \neq 0 \); this proves the first part of the lemma.
    
    The second part is proved by noticing first that the condition \( t_0 = \frac{2 \pi k_i}{\lambda_i} \) corresponds to lattices such that \( e^{tN_\lambda} = \text{Id} \) for any \( t \in \mathrm{T}(\Gamma) \). Therefore,
    \[
    e^{B_r} = S e^{t_0 N_\lambda} S^{-1} = \text{Id}.
    \]
    
    The latter equation, together with the fact that \( r_i \mid r_{i+1} \) applied in Condition \eqref{xi-condition}, gives the following property:
    \[
    \xi_0 = \left(\frac{z_1}{r_1}, \frac{z_2}{r_1}, \frac{z_3}{r_1 k_2}, \frac{z_4}{r_1 k_2}, \ldots, \frac{z_{2n-1}}{r_1 k_2 k_3 \cdots k_n}, \frac{z_{2n}}{r_1 k_2 k_3 \cdots k_n} \right),
    \]
    for some \( z_i \in \mathbb{N} \), and it can be verified that
    \[
    (0, \xi_0, 1)^{r_1 k_2 k_3 \cdots k_n} \in \mathbb{Q} \times \mathbb{Z}^{2n+1}.
    \]
    
    Since the \( 2n \)-components in \( \mathbb{R}^{2n} \) of the result are integers, every element of this form can be multiplied conveniently by \( (0, \pm e_i, 0) \in L(\xi_0) \) to obtain \( (q_1, 0, t_1) \in \Gamma \), for some \( q_x \in \mathbb{Q} \). Then for some integer \( y_1 \), \( (q_1, 0, t_1)^{y_1} = (y_1 q_1, 0, y_1 t_1) \in \mathbb{Z}^{2n+1} \).
    
    Finally, since \( (\pm 1, 0, 0) \in L(\xi_0) \), after convenient multiplications, one can construct an element \( (1, 0, k) \) in \( L(\xi_0) \) such that \( \Phi^{-1}(1, 0, k) = (w, 0, \widetilde{t_0} k) \).
\end{proof}

\begin{obs}\label{obs-osc}
    Let \( \Gamma \) be a lattice of the oscillator group \( \text{Osc}_n(\lambda_1, \ldots, \lambda_n) \) with \( t_0 = \frac{2\pi k_i}{K_0 \lambda_i} \) as in Equation \eqref{oscilator-N}. Notice that:
    \begin{itemize}
        \item The lightlike geodesics in \( \text{Osc}_n(\lambda_1, \ldots, \lambda_n) \) with \( a = 0 \) (see \eqref{geo2}), verify \( b_j = c_j = 0 \) for all \( j = 1, \ldots, n \). Consequently, they take the form \( \alpha_d(s) = (ds, 0, 0) \) and intersect \( \Gamma \) because there exists \( w > 0 \) such that \( (w, 0, 0) \in \Gamma \) according to the last lemma. This means that \( \alpha(\tilde{s}) = (w, 0, 0) \) for some \( \tilde{s} > 0 \).
        \item The lightlike geodesics with \( a \neq 0 \) verify \( \alpha\left(\frac{K_0 t_0}{a}\right) = (0, 0, K_0 t_0) \), see the expressions in \eqref{geo_osc_1}. If the lattice \( \Gamma \) contains an element of the form \( (0, 0, \hat{t}) \) with \( \hat{t} = p t_0 \) for some \( p \in \mathbb{Z} \), then
        \[
        \alpha(p K_0 t_0) = \alpha(K_0 t_0)^p = (0, 0, \hat{t})^p \in \Gamma.
        \]
    \end{itemize}
\end{obs}    

\begin{thm}\label{teoremaoscilador}
    Let \( \Gamma \) be a cocompact lattice of \( \text{Osc}_n(\lambda_1, \ldots, \lambda_n) \), and consider the compact Lorentzian manifold \( M = \text{Osc}_n(\lambda_1, \ldots, \lambda_n) / \Gamma \). Then only one of the following situations occurs:
    \begin{itemize}
        \item Either \( \Gamma \) contains an element of the form \( (0, 0, t_0) \), for some \( t_0 \in \mathbb{R}, t_0 \neq 0 \), and in this situation, every lightlike geodesic of \( M \) is closed.
        \item Or, for any \( t \neq 0 \), one has \( (0, 0, t) \notin \Gamma \). In this case, at every point in \( M \) there is exactly one direction for which all lightlike geodesics of \( M \) are closed, and for any other direction, they are non-closed. This direction is spanned by the lightlike element \( Z \in \mathfrak{osc}_n(\lambda_1, \ldots, \lambda_n) \).
    \end{itemize}
\end{thm}

\begin{proof}
    Recall that it suffices to study the geodesics starting at \( o := \pi(e) \) and that every geodesic \( \hat{\alpha} \) is the projection of some geodesic \( \alpha \) on \( \text{Osc}_n(\lambda_1, \ldots, \lambda_n) \): \( \hat{\alpha} = \pi(\alpha) \) with \( \alpha(0) = e \). Also, \( \hat{\alpha} \) is closed in \( M \) if \( \alpha(s) \in \Gamma \) for some \( s > 0 \).
    
    As observed in \ref{obs-osc}, all lightlike geodesics of the form \( \pi((ds, 0, 0)) \) are closed on the compact space \( M \), and so geodesics with this direction will always be closed. Therefore, to prove the theorem, it may hold that any lightlike geodesic with a different starting direction is either closed or simple.\footnote{Is this correct?}
    
    Let \( \alpha \) be a lightlike geodesic with a direction \( X \), which is linearly independent with \( Z \), and suppose that it is closed. This means that there exists some \( \gamma = (z, u, t) \in \Gamma \) for which \( \alpha(s) = \gamma \) for some \( s > 0 \). Since the curve \( \alpha \) is a one-parameter subgroup of \( \text{Osc}_n(\lambda_1, \ldots, \lambda_n) \), then for any integer \( m \): \( \alpha(ms) = \gamma^m \), which is an element of \( \Gamma \). Recall also that since \( t \in \mathrm{T}(\Gamma) \), it is of the form \( r t_0 \) for some integer \( r \) and \( t_0 = \frac{2 \pi k_i}{K_0 \lambda_i} \). Finally, since \( s = \frac{t}{a} \), one can compute that \( \gamma^{K_0} = \alpha\left(K_0 \frac{t}{a}\right) = (0, 0, K_0 t) = (0, 0, K_0 r t_0) \), and therefore, since \( K_0 r \) is an integer, this element is in the lattice and every lightlike geodesic of \( M \) is closed.
    
    In conclusion, when an element of the form \( (0, 0, k t_0) \) is in the lattice, every lightlike geodesic of \( M \) is closed; otherwise, \( \hat{\alpha}_d(s) = \pi(ds, 0, 0) \) are the only closed geodesics at \( \pi(e) \).
\end{proof}

\begin{exa}\label{Lattice4}
    Both situations stated in the above theorem are possible. Take, for instance, the three families of cocompact lattices constructed in \cite{BOV} for \( \text{Osc}_1(1) \),
    \begin{eqnarray*} \label{geodlight}
        \Lambda_{n,0} &=& \frac{1}{2n}\mathbb{Z} \times \mathbb{Z} \times \mathbb{Z} \times 2\pi \mathbb{Z},\\
        \Lambda_{n,\pi} &=& \frac{1}{2n}\mathbb{Z} \times \mathbb{Z} \times \mathbb{Z} \times \pi \mathbb{Z},\\
        \Lambda_{n,\frac{\pi}{2}} &=& \frac{1}{2n}\mathbb{Z} \times \mathbb{Z} \times \mathbb{Z} \times \frac{\pi}{2} \mathbb{Z},
    \end{eqnarray*}
    where \( n \in \mathbb{N} \), for which the authors proved that all lightlike geodesics of \( M_{n,0} = \text{Osc}_1(1)/\Lambda_{n,0} \), \( M_{n,\pi} = \text{Osc}_1(1)/\Lambda_{n,\pi} \), and \( M_{n,\pi/2} = \text{Osc}_1(1)/\Lambda_{n,\pi/2} \) are closed. However, other lattices can be obtained by noticing that
    \begin{eqnarray*}
        \phi_m &:& \text{Osc}_1(1) \rightarrow \text{Osc}_1(1),\\
        \phi_m(z, x, y, t) &=& (z + mt, x, y, t) \quad \text{for } m \in \mathbb{R},
    \end{eqnarray*}
    are automorphisms of \( \text{Osc}_1(1) \). So, the lattices \( \phi(\Lambda_{n,\bullet}) \) most likely do not contain an element of the form \( (0, 0, t) \). For example, given an integer \( p \neq 0 \), the lattice \( \phi_p(\Lambda_{n,0}) \) does not contain such an element since \( \frac{a}{2n} + p \cdot 2\pi b = 0 \) has no solution for integers \( a, b \). Thus, for these lattices, not every lightlike geodesic is closed.
\end{exa}

\begin{rem}
    Each lattice \( L(\xi_0) \) in Table 6 of \cite{MF} corresponds to a lattice \( \Gamma \) of \( \text{Osc}_1(1) \) where all the lightlike geodesics on \( \text{Osc}_1(1)/\Gamma \) are closed. Such correspondence is given by the inverse of the following group isomorphism \cite{MF}:
    \begin{eqnarray*}
        \phi: \text{Osc}_1(1) \rightarrow \text{Osc}_1(\omega_r, B_r), \\
        (z, v, t) \mapsto (r z, T_{x,y} v, t/\lambda),
    \end{eqnarray*}
    with
    \[
    T_{x,y} = \left( \begin{matrix}
        -\sqrt{y} & -\frac{x}{\sqrt{y}} \\
        0 & -\frac{1}{\sqrt{y}} \\
    \end{matrix} \right).
    \]
\end{rem}


To see the lightlike geodesics on \( \text{Osc}_1(1)/\Gamma \) for lattices \( \Gamma := \phi^{-1}(L(\xi_0)) \), one can first notice the following property:
\[
(0,0,\lambda k) = \phi^{-1}(0,0,k), \quad \forall k \in \mathbb{Z}.
\]
Therefore, if \( (0,0,k) \in L(\xi_0) \), the observation will be proved, according to Theorem 3.4.

\begin{rem}
    Prove that \( (0,0,k) \in L(\xi_0) \) for some \( k \in \mathbb{Z} \setminus \{ 0 \} \). Notice first that since \( \xi_0 \in \mathbb{Q}^2 \) \cite{MF}, we have \( (0, \xi_0, 1)^{y_1} \) for any \( y_1 \in \mathbb{Z} \). Additionally, since there exists \( N \) such that \( e^{N B_r} = \text{Id} \) (see derivation of Equation \eqref{oscilator-N}), one has \( (0, \xi_0, 1)^{N y_1 n_1} = (x_2, v_2, t_2) \in \mathbb{Z}^4 \) for some \( n_1 \in \mathbb{N} \). Finally,
    \[
    (x_2, v_2, t_2) \cdot (0, 1, 0)^{-v_2} \cdot (1, 0, 0)^{-x_1} = (0,0,t_2).
    \]
\end{rem}

To study timelike and spacelike geodesics on the compact spaces, one needs to consider the geodesics on \( \text{Osc}_n(\lambda_1, \ldots, \lambda_n) \) starting at the identity element as in (\ref{geo_osc_1}).

Let \( (d, b_j, c_j, a) \in \mathfrak{osc}_n \) be the initial velocity of a geodesic where \( a \neq 0 \), and let \( \hat{\gamma} = (\hat{z}, \hat{\eta}, \hat{t}) \) be an element of the lattice \( \Gamma \). Assume that \( \alpha(\hat{t}/a) = \gamma \) with \( \hat{t}/a > 0 \). In this situation, it holds that:
\begin{equation}\label{oscilador_geos_1}
    \left( \begin{matrix}
        \sin{\lambda_j \hat{t}} & \cos{\lambda_j \hat{t}} -1 \\
        1 - \cos{\lambda_j \hat{t}} & \sin{\lambda_j \hat{t}} \\
    \end{matrix} \right)
    \left( \begin{matrix}
        b_j \\
        c_j \\
    \end{matrix} \right) =
    \left( \begin{matrix}
        \hat{b_j} \\
        \hat{c_j}
    \end{matrix} \right),
\end{equation}

\begin{equation}\label{oscilador_geos_2}
    \hat{z} = \left(d + \frac{1}{2 a} \sum_{k=1}^{n} \frac{b_k^2 + c_k^2}{\lambda_k}\right) \frac{\hat{t}}{a} - \frac{1}{2 a^2} \sum_{k=1}^{n} \frac{b_j^2 + c_j^2}{\lambda_k^2} \sin(\lambda_k \hat{t}).
\end{equation}

These expressions are used to prove the first part of the following theorem.

\begin{thm}\label{othergeodesics}
    For any lattice \( \Gamma \) of \( \text{Osc}_n(\lambda_1, \ldots, \lambda_n) \), there are both closed and open timelike and spacelike geodesics on the compact space \( \text{Osc}_n(\lambda_1, \ldots, \lambda_n) / \Gamma \).
\end{thm}

		
\begin{proof}

	1) **Existence of closed timelike and spacelike geodesics:** As seen above, having closed timelike or spacelike geodesics of \( \text{Osc}_n(\lambda_1, \ldots, \lambda_n)/\Gamma \) is equivalent to having timelike or spacelike geodesics of \( \text{Osc}_n(\lambda_1, \ldots, \lambda_n) \) that intersect the lattice \( \Gamma \) at some positive time.
	
	Take an element \( (w,0,0) \in \Gamma \) with \( w>0 \) (see Lemma \ref{oscilador-elementos}), and consider any element \( \gamma = (z, \eta, t) \in \Gamma \). Thus, by multiplying those elements one gets  
	\[
	(w,0,0)^m \cdot (z, \eta, t) = (mw + z, \eta, t) \quad \text{for any } m \in \mathbb{Z}.
	\]
	
	Now consider the following two possibilities for \( \mathrm{K_0} \) (Equation \eqref{oscilator-N}):
	
	\begin{itemize}
		\item **Case \( \mathrm{K_0} = 1 \):** This is the case of the second item of Lemma \ref{oscilador-elementos}. Thus, there exists \( \gamma = (z, 0, t) \in \Gamma \) with \( z t \neq 0 \). Let \( \gamma_m := (mw + z, 0, t) = (w,0,0)^m \cdot (z,0,t) \) and consider the geodesic \( \alpha_m \) with initial velocity \( X = d Z + \sum_j (b_j X_j + c_j Y_j) + a T \) satisfying 
		\[
		a = t, \quad b_j = c_j = 0, \quad d_m = mw + z.
		\]
		It follows from the equations above that \( \alpha_m(1) = \gamma_m \) (in fact, for \( \mathrm{K_0} = 1 \), the matrix in \eqref{oscilador_geos_1} is trivial). Finally, \( \alpha_m \) is timelike or spacelike depending on whether \( \frac{mw + z}{t} \) is negative or positive, respectively; and either case can be achieved by choosing \( m \) conveniently.
	
		\item **Case \( \mathrm{K_0} > 1 \):** Consider \( \gamma = (x, u, (\mathrm{K_0}-1) t_0) \), and for \( m \in \mathbb{Z} \) define 
		\[
		\gamma_m := (mw + x, u, (\mathrm{K_0}-1) t_0) = (w,0,0)^m \cdot (x,u,(\mathrm{K_0}-1) t_0).
		\]
		For every \( \gamma_m \), the matrix in Equation \eqref{oscilador_geos_1} is non-singular (because if \( \lambda_j (\mathrm{K_0}-1) t_0 = 2\pi s_j \) for integers \( s_j \), one gets \( t_0 = \frac{2\pi (\mathrm{K_0}-1)}{\lambda_j} \), implying \( \mathrm{K_0} = 1 \), which is a contradiction). Therefore, Equation \eqref{oscilador_geos_1} gives unique solutions \( b_j, c_j \), independent of \( m \). Then, setting \( a = (\mathrm{K_0}-1) t_0 \) and solving Equation \eqref{oscilador_geos_2} for \( d = d_m \), one obtains parameters \( a, b_j, c_j, d_m \) such that \( \alpha_m(1) = \gamma_m \). Finally, these geodesics are closed in the quotient and are timelike or spacelike according to whether the expression
		\[
		2 p t_0 (mw + x) - \sum_{k=1}^{n} \frac{b_j^2 + c_j^2}{\lambda_k^2} \sin(\lambda_k (\mathrm{K_0}-1) t_0)
		\]
		is negative or positive, respectively. Both cases are achievable by choosing different values of \( m \).
	\end{itemize}
	
	2) **Existence of open timelike and open spacelike geodesics:**
	
	Consider the elements of the lattice of the form \( \hat{\gamma} = (\hat{z}, \hat{u}, p \mathrm{K_0} t_0) \in \Gamma \). These elements can be obtained by considering the \( \mathrm{K_0} \)th power of any element with a non-null \( t \)-component. Let \( \hat{s} \) such that \( \alpha(\hat{s}) = (z(\hat{s}), u(\hat{s}), t(\hat{s})) = \hat{\gamma} \), where \( \alpha \) is a geodesic of \( \text{Osc}_n(\lambda_1, \ldots, \lambda_n) \) (with \( a \neq 0 \)). Then it must be \( t(\hat{s}) = a \hat{s} = p \mathrm{K_0} t_0 \), which implies \( u(\hat{s}) = 0 \) and \( z(\hat{s}) = \left(d + \frac{1}{2a} \sum_{k=1}^{n} \frac{b_k^2 + c_k^2}{\lambda_k}\right) \frac{p \mathrm{K_0} t_0}{a} \), where \( a, b_k, c_k, d \) define the initial velocity of \( \alpha \).
	
	Let \( X = d' Z + \sum_j (b_j X_j + c_j Y_j) + a T \) be the initial condition of the geodesic \( \alpha_{d'} \). For any \( \varepsilon > 0 \), consider the interval \( I_d := [d, d + \varepsilon] \). Take \( d' \in I_d \) and assume that the geodesic \( \alpha_{d'} \) intersects the lattice at \( s' \), say \( \alpha_{d'}(s') \in \Gamma \). Then it must hold \( t'(s') = r' t_0 \) for some integer \( r' \). Now, define a function \( F: I_d \to \mathbb{Z} \), that sends \( d' \to r' \).
	
	Clearly, there exists an element denoted by \( r_{\infty} \in \mathbb{Z} \) such that the preimage \( F^{-1}(r_{\infty}) \) is an infinite set. Since \( F^{-1}(r_{\infty}) \subset I_d \), then this set is bounded, and it must contain a convergent sequence, namely \( \{d'_n\} \).
	
	Take the elements in the lattice \( \Gamma \) given by
	\begin{eqnarray*}
		\alpha_{d'_n}\left(\frac{r_{\infty} t_0}{a}\right) &=& \left(\left(d'_n + \frac{1}{2a} \sum_{k=1}^{n} \frac{b_k^2 + c_k^2}{\lambda_k}\right) \frac{\hat{t}}{a} - \frac{1}{2a^2} \sum_{k=1}^{n} \frac{b_j^2 + c_j^2}{\lambda_k^2} \sin(\lambda_k \hat{t})\right), \\ 
		&& R_1(r_{\infty} t_0) \left(\begin{matrix}
			b_1 \\
			c_1 \\
		\end{matrix}\right), \ldots, R_n(r_{\infty} t_0) \left(\begin{matrix}
			b_n \\
			c_n \\
		\end{matrix}\right), \quad r_{\infty} t_0 \right),
	\end{eqnarray*}
	with
	\[
	R_j(x) := \left(\begin{matrix}
		\sin(\lambda_j x) & \cos(\lambda_j x) - 1 \\
		1 - \cos(\lambda_j x) & \sin(\lambda_j x) \\
	\end{matrix}\right)
	\left(\begin{matrix}
		b_j \\
		c_j \\
	\end{matrix}\right),
	\]
	see Equation \eqref{oscilador_geos_1}. Since \( \{d'_n\}_n \) is convergent, the resulting sequence \( \{\alpha_{d'_n}\left(\frac{r_{\infty} t_0}{a}\right)\}_n \) also converges. This is a contradiction since \( \Gamma \) is discrete.
	
	\end{proof}
	
		
		
		
	
	\subsection{Remarks}
Consider the group \( G = \text{Osc}_n(\lambda_1, \ldots, \lambda_n) \times \mathbb{R} \). This Lie group is simply connected, has a bi-invariant Lorentzian metric, and admits cocompact lattices. However, its Lie algebra is not indecomposable. This fact affects the geometry of \( M = G/\Gamma \), where \( \Gamma \) is a cocompact lattice. Take, for instance, the lightlike geodesics of \( M \). Consider a geodesic of \( \mathbb{R} \) of the form \( \gamma(t) = \beta t \), and let \( \alpha \) be a geodesic of \( \text{Osc}_n(\lambda_1, \ldots, \lambda_n) \). Then the curve \( c(t) = (\alpha(t), \gamma(t)) \) is lightlike on \( M \) if the following equality holds:
\begin{equation}\label{remarkosc}
    0 = \, \langle \alpha'(0), \alpha'(0) \rangle \, + \, r^2,
\end{equation}
where \( \langle \,,\, \rangle \) is the metric of the oscillator (\ref{metricosc}) at the identity. It follows that \( \alpha \) must be a timelike geodesic of \( \text{Osc}_n(\lambda_1, \ldots, \lambda_n) \). Choose \( \Lambda_{n,0} \) as a lattice of \( \text{Osc}_1(1) \), and \( w \mathbb{Z} \) as a lattice of \( \mathbb{R} \), which holds for any real \( w \neq 0 \). Thus, \( \Lambda_{n,0} \times w \mathbb{Z} \) is a lattice of \( G = \text{Osc}_1(1) \times \mathbb{R} \).

The lightlike condition in this case, explicitly Equation (\ref{remarkosc}), gives
\[
0 = \, 2 a d + \frac{b + c}{2a} + \, r^2.
\]

Should \( c(t) \) be a closed lightlike geodesic of \( G \), then there exists some \( s > 0 \) such that \( \alpha(s) \in \Lambda_{n,0} \) and \( r(s) \in w \mathbb{Z} \). From the equations for \( \alpha \), it follows that \( s = \frac{2\pi k}{a} \) for some \( k \in \mathbb{Z} \) and \( z\left(\frac{2\pi k}{a}\right) = \left(d + \frac{b + c}{2a}\right) \frac{2\pi k}{a} = \frac{m}{2n} \) for some \( m \in \mathbb{Z} \). This can be reduced to
\[
r^2 = \, -\frac{a^2 m}{2\pi k}.
\]

Additionally, it must be that \( \gamma\left(\frac{2\pi k}{a}\right) = r \frac{2\pi k}{a} = wz \), leading to \( r^2 = \frac{a^2 z^2 w^2}{(2\pi k)^2} \). Then, since \( a \neq 0 \), one may have
\[
w^2 = -\frac{2\pi k m}{z^2}.
\]

In conclusion, since it is possible to choose \( w \) such that the equality above never holds for any \( k, z \in \mathbb{Z} \), lightlike geodesics of \( \text{Osc}_1(1) \times \mathbb{R}/\Gamma \) are never closed. Take, for instance, \( w = e \in \mathbb{R} \).
 
	
	
	
	
\section{Isometries of the Oscillator Groups and Compact Quotients}
In this section, we study the isometries of the oscillator groups and their compact quotients.

An isometry of a Lie group \( (G, \langle \,,\, \rangle) \) is a differentiable diffeomorphism \( \Psi: G \to G \) such that its differential preserves the metric at every point. The group of isometries of a Lie group with a left-invariant pseudo-Riemannian metric can be expressed as \( \text{Iso}(G) = L(G)F(G) \), where \( L(G) \) represents the subgroup of left-translations, and \( F(G) \) is the set of those isometries that fix the identity element of \( G \). Since every isometry \( \psi \) decomposes as \( \psi = L_g \circ \phi \), where \( \phi(e) = e \), the main question is to determine \( F(G) \).

For any \( \phi \in F(G) \), its differential \( d\phi_e \) is a linear map on \( \mathfrak{g} \). Let \( F(\mathfrak{g}) \) denote the set of \( d\phi_e \) for \( \phi \in F(G) \).

A local isometry is a map \( G \to G \) such that at the identity element \( e \), it is a local diffeomorphism \( \Psi': V_1 \to V_2 \), where \( V_1 \) and \( V_2 \) are neighborhoods of \( e \in G \), and the differential \( d\Psi' \) preserves the metric at every point of \( V_1 \). To compute local isometries, Müller proved the following result.

\begin{thm}[\cite{MU} Theorem 2.2] 
    Let \( (G, \langle \,,\, \rangle) \) denote a Lie group with a bi-invariant metric. Let \( A \) be a linear endomorphism of \( \mathfrak{g} \). Then there exists a local isometry \( \Phi \) of \( G \) at \( e \) such that \( d\Phi_e = A \) if and only if \( A \) satisfies the following conditions:
    \begin{enumerate}
        \item \( \langle AX, AY \rangle = \langle X, Y \rangle \) for all \( X, Y \in \mathfrak{g} \),
        \item \( A([X,[Y,Z]]) = [AX,[AY,AZ]] \) for all \( X, Y, Z \in \mathfrak{g} \).
    \end{enumerate}
\end{thm}

The aim now is to determine the isometry group of \( \text{Osc}_n(\lambda_1, \ldots, \lambda_n) \). As explained, the essential point is to determine the isotropy subgroup \( F(\text{Osc}_n(\lambda_1, \ldots, \lambda_n)) \). Define:

- \( \mathfrak{f}(\text{Osc}_n(\lambda_1, \ldots, \lambda_n)) \) as the Lie algebra of \( F(G) \),
- \( F(\mathfrak{osc}_n(\lambda_1, \ldots, \lambda_n)) \) as the group of isometries of the bilinear form on \( \mathfrak{osc}_n(\lambda_1, \ldots, \lambda_n) \), given by:
\[
Q_e = dtdz + \sum_{j=1}^n \frac{1}{\lambda_j}(dx_j^2 + dy_j^2),
\]
- \( \mathfrak{f}(\mathfrak{osc}_n(\lambda_1, \ldots, \lambda_n)) \) as the Lie algebra of \( F(\mathfrak{osc}_n(\lambda_1, \ldots, \lambda_n)) \).

Since \( \text{Osc}_n(\lambda_1, \ldots, \lambda_n) \) is simply connected, the following map is an isomorphism (see \cite{MU}):
\[
\phi \in F(\text{Osc}_n(\lambda_1, \ldots, \lambda_n)) \quad \mapsto \quad d\phi_e \in F(\mathfrak{g}).
\]
Moreover, the group \( F(\mathfrak{g}) \) consists of the linear maps \( A: \mathfrak{g} \to \mathfrak{g} \) satisfying the conditions of the theorem above.

Bourseau, in \cite{Bou}, studied the group \( F(\mathfrak{osc}_n(\lambda_1, \ldots, \lambda_n)) \). In the following paragraphs, we reproduce the main information from that work. Let \( \rho \) denote the following matrix of \( GL(2n, \mathbb{R}) \):
\[
\rho = \left( 
\begin{matrix} 
\lambda_1 & & & & \\
& \lambda_1 &   & &\\
& & \ddots & & \\
& & & \lambda_n & \\
& & & & \lambda_n
\end{matrix}\right).
\]
Let \( p \in \mathbb{N} \) with \( n_0 := 0 < n_1 < \cdots < n_p := n \) such that
\[
\rho_{\nu} := \lambda_{n_{\nu - 1} + 1} = \cdots = \lambda_{n_{\nu}} \quad \text{for } \nu = 1, \ldots, p,
\]
and let \( m_{\nu} := n_{\nu} - n_{\nu - 1} \).

\begin{prop}
    For the bi-invariant metric in \eqref{metricosc}, let \( A \in F(\mathfrak{g}) \). Then for \( \nu = 1, \ldots, p \), the map \( A \) has a matrix in the basis \( Z, \{X_i, Y_i\}, T \) of \( \mathfrak{osc}_n(\lambda_1, \ldots, \lambda_n) \):
    \[
    \varepsilon  \left( 
    \begin{matrix} 
        1 & c_1^{\tau} & \cdots & c_p^{\tau} & -\frac{1}{2} \sum_{\nu=1}^p \rho c_{\nu}^{\tau} c_{\nu} \\
        0 & B_1 & & & -\rho_1 B_1 c_1 \\
        \vdots & & \ddots & & \vdots \\
        0 & & & B_p & -\rho_p B_1 c_p \\
        0 & & & 0 & 1
    \end{matrix}\right),
    \]
    where \( \varepsilon = \pm 1 \), \( c_{\nu} \in \mathbb{R}^{2m_{\nu}} \), and \( B_{\nu} \in \mathrm{O}(2 m_{\nu}) \).
\end{prop}

Below, we describe the structure of \( F(\mathfrak{osc}_n(\lambda_1, \ldots, \lambda_n)) \). Recall that \( \mathrm{O}(1) = \{-1, 1\} \).
 

\begin{defn}
    Let \( K \) be the compact group
    \[
    K = \mathrm{O}(1) \times \prod_{\nu=1}^p \mathrm{O}(2m_{\nu}).
    \]
    Define the semidirect product
    \[
    F = K \ltimes_{\pi} \mathbb{R}^{2n},
    \]
    where
    \[
    \pi(\varepsilon, B_1, \ldots, B_p)(c) = \left( \begin{matrix}
    B_1  & & \\
     & \ddots & \\
      & & B_p
    \end{matrix} \right) c.
    \]	
\end{defn}

\begin{prop}
    The map \( \Psi: F(\mathfrak{osc}_n(\lambda_1, \ldots, \lambda_n)) \to F \) given by
    \[
    \Psi \left( \varepsilon  \left( 
    \begin{matrix} 
    1 & c_1^{\tau} & \cdots & c_p^{\tau} & -\frac{1}{2} \sum_{\nu=1}^p \rho c_{\nu}^{\tau} c_{\nu} \\
    0 & B_1 & & & -\rho_1 B_1 c_1 \\
    \vdots & & \ddots & & \vdots \\
    0 & & & B_p & -\rho_p B_1 c_p \\
    0 & & \cdots & 0 & 1
    \end{matrix}\right) \right) = (\varepsilon, B_1, \ldots, B_p, c)
    \]
    is an isomorphism of Lie groups.
\end{prop}

To describe the structure of \( F \), we identify the inner automorphisms, introduced as conjugation maps. Let \( I_h \) denote the inner automorphism, with \( d_e I_h = \text{Ad}(h): \mathfrak{osc}_n(\lambda_1, \ldots, \lambda_n) \). If we denote \( h = (z, v, t) \), clearly \( I_h = I_{(0, v, t)} \).

Let \( P_{\lambda_i}(t) \in \mathrm{SO}(2) \) be defined by
\[
P_{\lambda_i}(t) := \left\{
    \begin{array}{cl}
    \left( \begin{matrix}
    \sin(t\lambda_i) & 1 - \cos(t\lambda_i) \\
    -1 + \cos(t\lambda_i) & \sin(t\lambda_i)
    \end{matrix} \right) & \text{for } t \in \mathbb{R} \setminus \left\{\frac{2m\pi}{\lambda_i} \mid m \in \mathbb{Z}\right\}, \\
    \left( \begin{matrix}
    1 & 0 \\
    0 & 1
    \end{matrix} \right) & \text{for } t \in \left\{\frac{2m\pi}{\lambda_i} \mid m \in \mathbb{Z}\right\}.
    \end{array}
\right.
\]
Define \( P(t) \in \mathrm{SO}(2n) \) by
\[
P(t) = \left( \begin{matrix}
P_{\lambda_1}(t) & & \\
& \ddots & \\
& & P_{\lambda_n}(t)
\end{matrix} \right).
\]

\begin{lem}
    Consider the following element in \( F(\mathfrak{osc}_n(\lambda_1, \ldots, \lambda_n)) \):
    \[
    \left( \begin{matrix}
    1 & & \\
    & B & \\
    & & 1
    \end{matrix} \right).
    \]
    Then there exists an isometry \( \Theta(B): \text{Osc}_n(\lambda_1, \ldots, \lambda_n) \to \text{Osc}_n(\lambda_1, \ldots, \lambda_n) \) given by
    \[
    \Theta(B)(z, v, t) = (z, P(t)^{\tau} B P(t) v, t),
    \]
    with \( d \Theta(B)_e = \left( \begin{matrix}
    1 & & \\
    & B & \\
    & & 1
    \end{matrix} \right) \).
\end{lem}


Let \( K_1(\text{Osc}_n(\lambda_1, \ldots, \lambda_n)) \) be defined by
\[
K_1(\text{Osc}_n(\lambda_1, \ldots, \lambda_n)) = \{\Theta(B): \text{Osc}_n(\lambda_1, \ldots, \lambda_n) \to \text{Osc}_n(\lambda_1, \ldots, \lambda_n) \mid
\]
\[
\qquad \qquad \qquad \Theta(B)(z, v, t) = (z, P(t)^{\tau} B P(t) v, t), \, B_{\nu} \in \mathrm{O}(2m_{\nu})\},
\]
and let
\[
K(\text{Osc}_n(\lambda_1, \ldots, \lambda_n)) = K_1(\text{Osc}_n(\lambda_1, \ldots, \lambda_n)) \cup s K_1(\text{Osc}_n(\lambda_1, \ldots, \lambda_n)),
\]
where \( s: \text{Osc}_n(\lambda_1, \ldots, \lambda_n) \to \text{Osc}_n(\lambda_1, \ldots, \lambda_n) \), \( s(g) = g^{-1} \), is the inversion map.

Moreover,
\[
K_1(\text{Osc}_n(\lambda_1, \ldots, \lambda_n))_0 = \{\Theta(B) \in K_1(\text{Osc}_n(\lambda_1, \ldots, \lambda_n)) \mid B_{\nu} \in \mathrm{SO}(2m_{\nu})\}.
\]
Recall that the conjugation map \( I_{(z, v, t)} \) depends only on \( (v, t) \), and the set \( \{(v, t) \mid v \in \mathbb{R}^{2n}, t \in \mathbb{R}\} \) with the structure \( \text{Osc}_n(\lambda_1, \ldots, \lambda_n)/\{(z, 0, 0)\}_{z \in \mathbb{R}} \) is a solvable Lie group of dimension \( 2n+1 \). Denote by \( \text{Int}(\text{Osc}_n(\lambda_1, \ldots, \lambda_n)) \) the set of inner automorphisms, which is a subgroup of the isometry group.

Let \( s: \text{Osc}_n(\lambda_1, \ldots, \lambda_n) \to \text{Osc}_n(\lambda_1, \ldots, \lambda_n) \) denote the inversion isometry:
\[
s: (z, v, t) \quad \mapsto \quad (z, v, t)^{-1} = (-z, -R(-t)v, -t).
\]

\begin{thm}[\cite{Bou}] \label{Bou}
    The subgroup of isometries fixing the identity element has the following structure:
    \[
    F(\text{Osc}_n(\lambda_1, \ldots, \lambda_n)) = K(\text{Osc}_n(\lambda_1, \ldots, \lambda_n)) \cdot \text{Int}(\text{Osc}_n(\lambda_1, \ldots, \lambda_n)),
    \]
    with \( K(\text{Osc}_n(\lambda_1, \ldots, \lambda_n)) \cap \text{Int}(\text{Osc}_n(\lambda_1, \ldots, \lambda_n)) = \{ \text{id} \} \).
    
    Furthermore, \( \text{Int}(\text{Osc}_n(\lambda_1, \ldots, \lambda_n)) \) is a normal subgroup in \( F(\text{Osc}_n(\lambda_1, \ldots, \lambda_n)) \), and it holds:
    \[
    \begin{array}{crcl}
        (i) & \Theta(B) \circ I_{(v,t)} \circ \Theta(B)^{-1} & = & I_{(JBJ^{\tau}v, t)};\\
        (ii) & s \circ I_{(v,t)} \circ s^{-1} & = & I_{(v,t)};\\
        (iii) & s \circ \Theta(B) \circ s^{-1} & = & \Theta(B).
    \end{array}
    \]
    \( F(\text{Osc}_n(\lambda_1, \ldots, \lambda_n)) \) consists of \( 2^{p+1} \) connected components, and for the connected component of the identity, one has
    \[
    F(\text{Osc}_n(\lambda_1, \ldots, \lambda_n))_0 = K_1(\text{Osc}_n(\lambda_1, \ldots, \lambda_n))_0 \cdot \text{Int}(\text{Osc}_n(\lambda_1, \ldots, \lambda_n)).
    \]
\end{thm}

As a corollary, the Lie algebra of \( F(\text{Osc}_n(\lambda_1, \ldots, \lambda_n)) \) is isomorphic to the following one:
\[
\mathfrak{f} = \prod_{\nu=1}^p \mathfrak{so}(2m_{\nu}) \ltimes \mathbb{R}^{2n}.
\]
Let \( \text{Aut}(\text{Osc}_n(\lambda_1, \ldots, \lambda_n)) \) denote the group of automorphisms of the oscillator group, and let \( \text{Sp}(2m_{\nu}) \) denote the group of \( 2m_{\nu} \times 2m_{\nu} \)-symplectic matrices over \( \mathbb{R} \). Now, we determine which isometries are automorphisms. Denote by \( \tilde{K}_1 \) the compact subgroup of \( K_1(\text{Osc}_n(\lambda_1, \ldots, \lambda_n)) \) given by 
\[
\tilde{K}_1 = \{\Theta(B) \in K_1(\text{Osc}_n(\lambda_1, \ldots, \lambda_n)) \mid B_{\nu} \in \mathrm{O}(2m_{\nu}) \cap \text{Sp}(2m_{\nu})\}.
\]
Take \( M \in \mathrm{O}(2n) \) given by
\[
M = \left( \begin{matrix}
1 & 0 & & & \\
0 & -1 & & & \\
& & \ddots & & \\
& & & 1 & 0 \\
& & & 0 & -1
\end{matrix}
\right).
\]

\begin{prop}
    Let \( \tilde{K} \) be the compact subgroup of \( K(\text{Osc}_n(\lambda_1, \ldots, \lambda_n)) \) given by
    \[
    \tilde{K} = \tilde{K}_1 \cup s \circ \Theta(M) \circ \tilde{K}_1,
    \]
    then it holds
    \[
    F(\text{Osc}_n(\lambda_1, \ldots, \lambda_n)) \cap \text{Aut}(\text{Osc}_n(\lambda_1, \ldots, \lambda_n)) = \tilde{K} \cdot \text{Int}(\text{Osc}_n(\lambda_1, \ldots, \lambda_n)).
    \]
\end{prop}

Finally, Bourseau studied the isometry group of \( \text{Osc}_n(\lambda_1, \ldots, \lambda_n) \). He found that the Lie algebra of \( \text{Iso}(\text{Osc}_n(\lambda_1, \ldots, \lambda_n)) \) is the semidirect product
\[
\mathfrak{iso}(\text{Osc}_n(\lambda_1, \ldots, \lambda_n)) = \left(\prod_{\nu=1}^p \mathfrak{so}(2m_{\nu})\right) \ltimes \mathfrak{g}_{2n},
\]
where \( \mathfrak{g}_{2n} \) is the oscillator algebra of dimension \( 4n+2 \).

Thus, the isometry group follows as
\[
\text{Iso}(\text{Osc}_n(\lambda_1, \ldots, \lambda_n)) = L(\text{Osc}_n(\lambda_1, \ldots, \lambda_n)) \cdot F(\text{Osc}_n(\lambda_1, \ldots, \lambda_n)),
\]
and it holds that \( L(\text{Osc}_n(\lambda_1, \ldots, \lambda_n)) \cap F(\text{Osc}_n(\lambda_1, \ldots, \lambda_n)) = \{ \text{id} \} \).



\begin{rem}
    Note that since the inversion map \( h \mapsto h^{-1} \) is an isometry of the Lie group \( (\text{Osc}_n(\lambda_1, \ldots, \lambda_n), \lela \,,\, \rira) \), the compact spaces \( \Lambda \backslash \text{Osc}_n(\lambda_1, \ldots, \lambda_n) \) and \( \text{Osc}_n(\lambda_1, \ldots, \lambda_n)/\Lambda \) are isometric. In fact, the map \( x\Lambda \mapsto \Lambda x^{-1} \) is an isometry between both spaces. Clearly, \( \text{Osc}_n(\lambda_1, \ldots, \lambda_n) \) acts on \( \text{Osc}_n(\lambda_1, \ldots, \lambda_n)/\Lambda \) on the left transitively.
\end{rem}

\subsection{Isometries in the quotients.}

The aim now is to study the isometry group of the quotient spaces. Let \( \Lambda \) denote a discrete subgroup of \( \text{Osc}_n(\lambda_1, \ldots, \lambda_n) \) such that \( M = \text{Osc}_n(\lambda_1, \ldots, \lambda_n)/\Lambda \) is a compact space. Since the metric on \( \text{Osc}_n(\lambda_1, \ldots, \lambda_n) \) is both right- and left-invariant, it can be induced to the cosets \( g\Lambda \in M \). Indeed, an isometry of \( M \) gives rise to a local isometry in \( \text{Osc}_n(\lambda_1, \ldots, \lambda_n) \), since the projection \( \text{Osc}_n(\lambda_1, \ldots, \lambda_n) \to M \) is a submersion and a local isometry. Thus, one has a local isometry of \( \text{Osc}_n(\lambda_1, \ldots, \lambda_n) \) satisfying conditions in the paragraphs above.

On the other hand, some isometries of \( \text{Osc}_n(\lambda_1, \ldots, \lambda_n) \) can be induced to the quotient. The next result specifies the conditions for such maps.

\begin{defn}
    Let \( f \) be an isometry of \( \text{Osc}_n(\lambda_1, \ldots, \lambda_n) \), and \( \Lambda \) a lattice. We say that \( f \) is \emph{fiber preserving} if \( f(g)^{-1} f(g\lambda) \in \Lambda \) for all \( g \in \text{Osc}_n(\lambda_1, \ldots, \lambda_n) \) and for every \( \lambda \in \Lambda \).
\end{defn}

If \( f \) is a fiber-preserving isometry, it induces an isometry \( \tilde{f} \) on the compact space \( M = \text{Osc}_n(\lambda_1, \ldots, \lambda_n)/\Lambda \) by defining \( \tilde{f}(g\Lambda) = f(g)\Lambda \).

\begin{obs}
    Note the following facts:
    \begin{itemize}
        \item Translations on the left by elements of the group are fiber-preserving maps. Every map \( L_h \) induces the isometry \( \tau_h \) in \( M \). In particular, \( L_{\lambda}(\Lambda) \subseteq \Lambda \) for every \( \lambda \in \Lambda \). Denote by \( \tilde{L}(M) = \{\tau_g : g \in \text{Osc}_n(\lambda_1, \ldots, \lambda_n)\} \).
        \item If \( f \) is a fiber-preserving map that fixes the identity, then \( f(\lambda) \in \Lambda \) for every \( \lambda \) in the lattice \( \Lambda \).
    \end{itemize}
\end{obs}

Recall that whenever the lattices \( \Lambda_1 \) and \( \Lambda_2 \) are not pairwise isomorphic, they determine non-diffeomorphic solvmanifolds (see, for instance, \cite{Ra}).

One can study the isometry group of \( \text{Osc}_n(\lambda_1, \ldots, \lambda_n)/\Lambda \) once one has information about the isometry group of \( \text{Osc}_n(\lambda_1, \ldots, \lambda_n) \). The following result is a consequence of the Lifting theorem. The proof can be seen in \cite{BOV}.

\begin{thm}
    Let \( G \) be an arcwise-connected, simply connected Lie group with a bi-invariant metric, and let \( \Lambda \) be a discrete subgroup of \( G \). Then every isometry \( f \) of \( G/\Lambda \) is induced by a fiber-preserving isometry of \( G \).
\end{thm}

In view of this, we proceed to study the fiber-preserving isometries of \( G \), specifically, those in the isotropy subgroup.

Analogously to \cite{BOV}, computations show that neither the inversion map \( s \) nor the map \( \Theta(B) \) are fiber-preserving. In fact, to see this, assume \( \lambda = (\tilde{z}, \tilde{v}, \tilde{t}) \in \Lambda \). By computing 
\[
(z, v, t)(\tilde{z}, \tilde{v}, \tilde{t})^{-1}(-z, -R(-t)v, -t)
\]
and looking at the component in \( \mathbb{R}^{2n} \), one obtains 
\[
v - R(-\tilde{t})v - R(t-t_1)\tilde{v},
\]
which must belong to \( \Lambda \cap \mathbb{R}^{2n} \) for every \( v \in \mathbb{R}^{2n} \) and \( t \in \mathbb{R} \). In particular, for \( v = 0 \), one gets that for all \( t \in \mathbb{R} \), it holds \( R(t-t_1)\tilde{v} \in \Lambda \cap \mathbb{R}^{2n} \subset \Lambda \), which is a countable set. This is a contradiction, and similarly for \( \Theta(B) \).

Thus, it remains to determine which inner automorphisms are fiber-preserving. Let \( I_h: \text{Osc}_n(\lambda_1, \ldots, \lambda_n) \to \text{Osc}_n(\lambda_1, \ldots, \lambda_n) \) be the conjugation map, then
\[
I_h(g)^{-1}I_h(g\lambda) = hg^{-1}h^{-1} hg\lambda h^{-1} = h\lambda h^{-1} \in \Lambda,
\]
for every \( \lambda \in \Lambda \). The condition above says that  
\( h \in N_G(\Lambda) \), the normalizer of the lattice \( \Lambda \) in \( \text{Osc}_n(\lambda_1, \ldots, \lambda_n) \).

Since any isometry \( f \) of the Lie group can be written as \( f = L_p \circ g \) with \( g \) an isometry fixing the identity element, we have that any isometry in the quotient space \( M = \text{Osc}_n(\lambda_1, \ldots, \lambda_n)/\Lambda \) can be written as \( \tilde{f} = \tau_g \circ \tilde{h} \), where \( \tilde{h} \) denotes the isometry induced to the quotient by \( h \), \( \tilde{h}(g\Lambda) = h(g)\Lambda \).
 



Consider the following homomorphisms where \( G = \text{Osc}_n(\lambda_1, \ldots, \lambda_n) \):
\begin{itemize}
    \item \( \widetilde{I}: N_G(\Lambda) \to \text{Iso}(M) \) given by \( \widetilde{I}(h) = \widetilde{I}_h \), and
    \item \( \tau: G \to \text{Iso}(M) \), which gives \( \tau(g) = \tau_g \).
\end{itemize}

By the Isomorphism Theorem, one has \( \tilde{L}(M) = \text{Im}\,\tau = G/\ker\tau \), and \( \ker\tau \) contains the elements in the intersection of the center and \( \Lambda \): \( Z(\text{Osc}_n(\lambda_1, \ldots, \lambda_n)) \cap \Lambda \), implying that \( \tau \) is not injective.

On the other hand, it is easy to see that for any \( h \in Z(\text{Osc}_n(\lambda_1, \ldots, \lambda_n)) \), one has \( I_h(x) = x \), so that \( h \in \ker \widetilde{I} \). In this case, \( \widetilde{I} \) is not injective.

In any case, to specify those statements, one needs more information about \( \Lambda \).

\begin{prop}
    Let \( \Lambda \) be a lattice in \( \text{Osc}_n(\lambda_1, \ldots, \lambda_n) \). The isometries in the Lie group that are fiber-preserving correspond to translations on the left by elements of the group and the inner automorphisms \( I_h \) for \( h \in N_G(\Lambda) \). Moreover, any isometry \( f \) in \( M = \text{Osc}_n(\lambda_1, \ldots, \lambda_n)/\Lambda \) can be written as \( f = \tau_g \circ \widetilde{I}_h \), but this is not necessarily unique.
\end{prop}

\begin{rem}
    For the subgroups \( \Lambda \) in Example \ref{Lattice4}, it was proved in \cite{BOV} that
    \[
    \tilde{L}(M) \cap \text{Im}\, \tilde{I} = \{\tau_Z \circ \widetilde{I}_{\lambda}, \, \text{for } Z \in Z(G), \, \lambda \in \Lambda_{k,s}\}.
    \]
\end{rem}

\begin{exa}
    In dimension four, one may consider the lattices \( \Lambda_{k,0}, \Lambda_{k,\pi}, \Lambda_{k,\pi/2} \) in the oscillator group \( G \) of dimension four with \( \lambda_1 = 1 \). See Example \ref{Lattice4}. As stated in \cite{BOV}, since the subgroups \( \Lambda_{k,j} \) are not pairwise isomorphic, the corresponding compact spaces are not homeomorphic.

    To compute the normalizers of such lattices, one may find the elements \( (z,v,t) \) such that
    \[
    (z,v,t)(\tilde{z},\tilde{v},\tilde{t})(-z,-R(-t)v,-t) \in \Lambda,
    \]
    for all \( (\tilde{z},\tilde{v},\tilde{t}) \in \Lambda \), where \( \Lambda \) is a lattice. The proof follows by writing down the coordinates.
    \begin{itemize}
        \item For \( \Lambda_{k,0} \), the map \( R(nt_0) \) is the identity for \( t_0 = 2\pi \). Thus, one has \( v + R(t)\tilde{v} - R(\tilde{t})v \in \mathbb{Z}^2 \), which implies that \( R(t) = s \frac{\pi}{2} \) for \( s \in \mathbb{Z} \).

        From the \( z \)-coordinate, one has

        \[
        \tilde{z} + \frac{1}{2} v^{\tau} J R(t)\tilde{v} - \frac{1}{2} v^{\tau} J R(\tilde{t})v - (R(t)\tilde{v})^{\tau} J R(\tilde{t})v = \frac{u}{2k},
        \]
        for some \( u \in \mathbb{Z} \),

        which implies \( v^{\tau} J R(t)\tilde{v} = \frac{\tilde{u}}{2k} \) for some \( \tilde{u} \in \mathbb{Z} \). Therefore, \( v \in \frac{1}{2k}\mathbb{Z}^2 \).

        So, we get \( N_G(\Lambda_{k,0}) = \mathbb{R} \times \frac{1}{2k}\mathbb{Z}^2 \times \frac{\pi}{2}\mathbb{Z} \).
        \item For \( \Lambda_{k,\pi} \), the map \( R(2n\pi) \) is the identity or \( R(n\pi) = -\text{Id} \) for \( n \) odd.

        A similar reasoning as above gives \( N_G(\Lambda_{k,\pi}) = \mathbb{R} \times \frac{1}{2}(\mathbb{Z})^2 \times \frac{\pi}{2}\mathbb{Z} \).
        \item For \( \Lambda_{k,\pi/2} \), the map \( R(n\pi/2) = \pm \text{Id} \) if \( n \) is even, with \( -\text{Id} \) if \( n \equiv 2 \mod(4) \). Thus, reasoning as above shows that \( N_G(\Lambda_{k,\pi/2}) = \mathbb{R} \times \mathbb{Z}^2 \times \frac{\pi}{2}\mathbb{Z} \).
    \end{itemize}

    In dimension six, the situation is much more complicated, as we show below.
\end{exa}

\subsection{An example in dimension six.}

Assume we have the Lie group \( \text{Osc}(1,\lambda) \), which has the differentiable structure of \( \mathbb{R}^6 \). As said in Lemma \ref{lema_medina}, to have a cocompact lattice, one needs that the real numbers \( 1,\lambda \) generate a discrete subgroup of \( \mathbb{R} \).

On the other hand, it is known that a subgroup \( H \) of \( \mathbb{R} \) is either discrete or dense. Moreover, if it is discrete, then \( H = \mathbb{Z}r \), where \( r = \inf(H \cap \mathbb{R}_{>0}) > 0 \). Thus, we are in the latter situation. This implies that there exists \( n \in \mathbb{Z} \) such that \( 1 = nr \), meaning that \( r \in \mathbb{Q} \). Analogously, since there exists \( s \in \mathbb{Z} \) such that \( sr = \lambda \), we have \( \lambda \in \mathbb{Q} \).

Thus, for the corresponding Lie group, we have, for some \( r \in \mathbb{Q} \), the map \( R(t) \) has a matrix presentation as follows:
\[
R(t) = \left( \begin{matrix} 
\cos(t) & -\sin(t) & 0 & 0 \\
\sin(t) & \cos(t) & 0 & 0 \\
0 & 0 & \cos(rt) & -\sin(rt) \\
0 & 0 & \sin(rt) & \cos(rt)
\end{matrix}
\right)
\]
 
Note that if \( r = p/q \) with \( p \) and \( q \) coprime numbers, then \( t_0 = q \) generates a subgroup of \( \mathbb{Z} \), and both \( \cos(2sq\pi)(m), \sin(2sq\pi)(m) \in \mathbb{Z} \) for every \( s, m \in \mathbb{Z} \), and also \( \cos(2sp\pi)(m), \sin(2sp\pi)(m) \in \mathbb{Z} \) for all \( s, m \in \mathbb{Z} \).

Thus, \( R(2st_0\pi)(\mathbb{Z} \times \mathbb{Z} \times \mathbb{Z} \times \mathbb{Z}) \subseteq \mathbb{Z} \times \mathbb{Z} \times \mathbb{Z} \times \mathbb{Z} \), which says that \( \Lambda_{k,0} = \frac{1}{2k} \mathbb{Z} \times \mathbb{Z}^4 \times 2q\pi\mathbb{Z} \) is a cocompact lattice of \( \text{Osc}(1,p/q) \) for \( k \in \mathbb{N} \).

Analogously, one proves that the following set is also a cocompact lattice:

\[
\Lambda_{k,\pi} = \frac{1}{2k} \mathbb{Z} \times \mathbb{Z}^4 \times q\pi\mathbb{Z}.
\]

In this case, \( \cos(sq\pi)(m) = \pm m, \sin(sq\pi)(m) = \pm m \), depending on the parity of \( sq \), and \( \cos(2sp\pi)(m) = \pm m, \sin(2sp\pi)(m) = \pm m \), depending on the parity of \( sp \). In any case, \( \cos(2sq\pi)(m), \sin(2sq\pi)(m) \in \mathbb{Z} \). A similar reasoning applies to the lattice

\[
\Lambda_{k,\pi/2} = \frac{1}{2k} \mathbb{Z} \times \mathbb{Z}^4 \times \frac{q\pi}{2}\mathbb{Z}.
\]

In this way, we generalize the lattices considered in \cite{BOV} for every \( k \in \mathbb{N} \) to 
\[
\Lambda_{k,q,M} = \frac{1}{2 k} \Z \times \Z^4 \times \frac{2 \pi q}{M} \Z, \quad \text{for } M = 1,2,4.
\]

Note that one obtains three distinct families for \( k, q \in \NN \) when:
\begin{enumerate}
    \item \( \forall q, k \), when \( M = 1 \),
    \item \( q \) odd, when \( M = 2 \),
    \item \( q \) odd, when \( M = 4 \).
\end{enumerate}

 

Given $g=(z,v,t) \in \text{Osc}_2(1,p/q)$ and $\lambda=(a,\eta,b) \in \Gamma$ for any of the lattices above, computing $g \lambda g^{-1}$ gives
\begin{equation}
    (a + \frac{1}{2}v^T J R(t) \eta - \frac{1}{2} v^T J R(b) v - \frac{1}{2} \eta^T R(t)^T J R(b) v, v + R(t) \eta - R(b) v, b).
\end{equation}

In order to have $g \Lambda g^{-1} \subset \Lambda$, one obtains the conditions on $z,v,t$ as follows:

\begin{align}
    v + R(t) \eta - R\left(\frac{2 \pi q}{M} c\right)v &\in \mathbb{Z}^4,   \label{c1} \\ 
    \frac{1}{2} \left[ v^T J R(t) \eta - v^T J R(b) v - \eta^T R(t)^T J R\left(\frac{2 \pi q}{M} c\right) v \right] &\in \frac{\mathbb{Z}}{2k}, \label{c2}
\end{align}
for any \( \eta \in \mathbb{R}^4 \) and \( b \in \frac{2 \pi q}{M} \mathbb{Z} \) with \( c = 0, \ldots, M-1 \).

For the case \( c = 0 \) in Equation \eqref{c1}, this leads to \( R(t) \eta \in \mathbb{Z}^4 \), and consequently 
\begin{equation}
    t \in \frac{q \pi}{2} \mathbb{Z}. \label{cc1}
\end{equation}

Similarly, \( c = 0 \) in Equation \eqref{c2} implies that 
\[ v^T J R(t) \eta \in \frac{\mathbb{Z}}{2k} \,\, \forall \eta \in \mathbb{Z}^4, \]     
and consequently 
\begin{equation}
    v \in \frac{\mathbb{Z}^4}{2k}. \label{cc2}
\end{equation}

For \( c = 1, \ldots, M-1 \), Equations \eqref{c1} and \eqref{c2} reduce to

\begin{align}
    v - R\left(\frac{2 \pi q}{M}c\right) v &\in \mathbb{Z}^4,  \label{ccc1} \\
    (v^T + v^T R^T\left(\frac{2 \pi q}{M}c\right)) J R(t) \eta - v^T J R\left(\frac{2 \pi q}{M}c\right) v  &\in \frac{\mathbb{Z}}{k}. \nonumber
\end{align}

The latter equation can be simplified by noticing that \( \eta = 0 \) implies 
\begin{equation}
    v^T J R\left(\frac{2 \pi q}{M}c\right)v \in \frac{\mathbb{Z}}{k}, \label{ccc2}
\end{equation}
and therefore \( (v^T + v^T R^T\left(\frac{2 \pi q}{M}c\right)) J R(t) \eta \in \frac{\mathbb{Z}}{k} \), 
which is equivalent to:

\begin{equation}
    v + R\left(\frac{2 \pi q}{M}c\right) v \in \frac{\mathbb{Z}^4}{k}. \label{ccc3}
\end{equation}

Consider the following set definitions, which are useful for the next lemma and proposition:
\[
\mathcal{C} := \{ (p,q,k,M): p,q,k \in \mathbb{N}, \text{ p and q coprime}, M \in \{ 1,2,4\}, \text{ q odd when } M > 1 \},
\]
\[
F := \frac{1}{2}\{ z \in \mathbb{Z} : z \text{ odd} \},
\]
\[
I_2 := \mathbb{Z}^2 \cup F^2,
\]
\[
I_4 := \mathbb{Z}^4 \cup F^4.
\]




\begin{lem}
	Let $(p,q,k,M) \in \mathcal{C}$, then $(z, (v_1,v_2,v_3,v_4),t)$ is an element of the normalizer of $\Lambda_{k,q,M}$ in $Osc(1,p/q)$ if it satisfies the following conditions:

   \def\arraystretch{1.5}
   \begin{center}
   \begin{tabular}{|c|c|}\hline
   \multirow{2}{*}{M=1,2,4} & $t \in \frac{q \pi}{2} \mathbb{Z}$   \\\cline{2-2}
	   & $(v_1,v_2,v_3,v_4) \in \mathbb{Z}^4 / {2 k}$  \\\hline
   \multirow{2}{*}{M=2} & $(v_1,v_2) \in \mathbb{Z}^2/2$  \\\cline{2-2}
	   & $(v_3,v_4) \in \mathbb{Z}^2/2$ if $p$ is odd   \\\hline
   \multirow{4}{*}{M=4} & $(v_1,v_2) \in I_2$  \\\cline{2-2}
	   & $(v_1,v_2) \in \mathbb{Z}^2 \text{ if } p \text{ is even and } k \text{ is odd}$  \\\cline{2-2}
	   & $(v_3,v_4) \in I_2 \text{ if } p \text{ is odd}$  \\\cline{2-2}
	   & $(v_1,v_2,v_3,v_4) \in I_4 \text{ if } p \text{ is odd and } k \text{ is odd}$  \\\hline
   \end{tabular}
   \end{center}
   
\end{lem}

\begin{proof}
   As seen above, given $(z,v,t)$ in the normalizer, then $v$ must satisfy Equations \eqref{cc1} and \eqref{cc2} for any $M=1,2,4$ and Equations \eqref{ccc1}, \eqref{ccc2}, and \eqref{ccc3} for $M=2,4$. For this proof, let $A=(v_1,v_2)$ and $B=(v_3,v_4)$.\\

   For $M=2$ and $M=4$, notice that the subspaces $\mathbb{R}^2 \times \{(0,0)\}$ and $\{(0,0)\} \times \mathbb{R}^2$ are invariant under the linear operators $J$ and $R\left(\frac{2 \pi q}{M}c\right)$, and therefore the linear equations \eqref{ccc1} and \eqref{ccc3} result equivalent to 

   \begin{align}
   A& - r\left(\frac{2 \pi q}{M}c\right) A \in \mathbb{Z}^2, \label{ab1}\\
   B& - r\left(\frac{2 \pi p}{M}c\right) B \in \mathbb{Z}^2, \label{ab2}\\
   A& + r\left(\frac{2 \pi q}{M}c\right) A \in \mathbb{Z}^2/k, \label{ab3}\\
   B& + r\left(\frac{2 \pi p}{M}c\right) B \in \mathbb{Z}^2/k, \label{ab4}
   \end{align}

where $r(t) := \begin{pmatrix} 
\cos(t) & -\sin(t)\\
\sin(t) & \cos(t)
\end{pmatrix}$. With this in mind and recalling from Section \ref{section1} that $J_1 = \begin{pmatrix} 
0 & -1\\
1 & 0
\end{pmatrix}$, Equation \eqref{ccc2} can be written as follows:

\begin{equation}
   A^{\tau} J_1 r\left(\frac{2 \pi q}{M}c\right) A + B^{\tau} J_1 r\left(\frac{2 \pi p}{M}c\right) B \in \mathbb{Z}/k. \label{ab5}
\end{equation}

\underline{For $(M=2, c=1)$ and $(M=4, c=2)$}: given that $r\left(\frac{2 \pi n}{M}c\right) = \pm \text{Id}$, Equations \eqref{ab1} to \eqref{ab4} result in:


\begin{align}
   2A& \in \mathbb{Z}^2 \text{ if } q \text{ is odd}, \label{ab6}\\
   2B& \in \mathbb{Z}^2 \text{ if } p \text{ is odd}, \label{ab7}\\
   2A& \in \mathbb{Z}^2/k \text{ if } q \text{ is even}, \nonumber \\
   2B& \in \mathbb{Z}^2/k \text{ if } p \text{ is even}. \nonumber
\end{align}

Notice that the last two equations are already satisfied by the condition $v \in \mathbb{Z}^4/{2k}$. Notice also that Condition \eqref{ab5} is trivial since the expression is null.\\


\underline{For $M=4, c=1,3$}: Equations \eqref{ab1} to \eqref{ab4} can be expressed as follows:

\begin{align}
   A& \pm J_1 A \in \mathbb{Z}^2 \text{ if } q \text{ is odd}, \label{ab8}\\
   B& \pm J_1 B \in \mathbb{Z}^2 \text{ if } p \text{ is odd}, \label{ab9}\\
   A& \pm J_1 A \in \mathbb{Z}^2/k \text{ if } q \text{ is odd}, \nonumber \\
   B& \pm J_1 B \in \mathbb{Z}^2/k \text{ if } p \text{ is odd}. \nonumber
\end{align}

The last two equations are weaker than the first two, so only \eqref{ab8} and \eqref{ab9} need to be considered. When condition \eqref{ab8} applies, condition \eqref{ab6} also applies, and the solution is $A \in I_2$. Analogously with \eqref{ab7} and \eqref{ab9}, the solution is $B \in I_2$.

It remains to add the conditions that result from \eqref{ab5}. From now on, assume $q$ is odd. Observe that only for $M=4, c=1,3$ the equation is not trivial. Consider first $p$ even, the condition can be expressed as
$$ v_1^2 + v_2^2 \in \mathbb{Z}/k. $$

Since $(v_1,v_2) \in  I_2$, the condition above is valid always when $k$ is even, while if $k$ is odd, $(v_1,v_2) \in \mathbb{Z}^2$ is required. 

When $p$ is odd, the condition can be expressed as one of the following two forms (depending on the precise values of $p$ and $q$):

$$ v_1^2 + v_2^2 \pm (v_3^2 + v_4^2) \in \mathbb{Z}/k.$$

Similar to the previous case, if $k$ is even, the condition holds since $(v_1,v_2), (v_3,v_4) \in I_2$. If $k$ is odd, $v \in I_4$ is required.

\end{proof}



\begin{prop}
    The normalizers of the lattices $\Lambda_{k,q,M} \subset Osc(1,p/q)$, for $(p,q,k,M) \in \mathcal{C}$, are given in the following table:
\begin{center}
\def\arraystretch{1.5}
\begin{tabular}{ |c|c|c| } 
\hline
$M$ & Conditions & Normalizer = $N(\Lambda_{k,q,M})$ \\
\hline
\multirow{1}{1em}{1} & - & $\mathbb{R} \times \mathbb{Z}^4 / {2k} \times q \frac{\pi}{2} \mathbb{Z}$ \\ 
\hline
\multirow{2}{1em}{2} & $p$ even & $\mathbb{R} \times \mathbb{Z}^2/2 \times \mathbb{Z}^2 / {2k} \times q \frac{\pi}{2} \mathbb{Z}$ \\ 
&  $p$ odd & $\mathbb{R} \times \mathbb{Z}^4/2 \times q \frac{\pi}{2} \mathbb{Z}$ \\ 
\hline
\multirow{4}{1em}{4} & $p$ even, $k$ even & $\mathbb{R} \times I_2 \times \mathbb{Z}^2 / {2k} \times q \frac{\pi}{2} \mathbb{Z}$ \\ 
&  $p$ even, $k$ odd & $\mathbb{R} \times \mathbb{Z}^2 \times \mathbb{Z}^2 / {2k} \times q \frac{\pi}{2} \mathbb{Z}$ \\ 
&  $p$ odd, $k$ even & $\mathbb{R} \times I_2 \times I_2 \times q \frac{\pi}{2} \mathbb{Z}$\\ 
&  $p$ odd, $k$ odd & $\mathbb{R} \times I_4  \times q \frac{\pi}{2} \mathbb{Z}$ \\ 
\hline
\end{tabular}
\end{center}
\end{prop}

This result is straightforward from the previous Lemma when combining all conditions.


  
    
\begin{thebibliography}{GGGG}
		
	\bibitem{BG} {\sc O. Baues, W. Globke}, {\it Rigidity of compact pseudo-Riemannian homogeneous spaces for solvable Lie groups}. 
		Int. Math. Res. Not. {\bf  2018} (10), 3199--3223 (2018). 
	\bibitem{Be} {\sc A. Beardon}, {\it The geometry of discrete groups}. Springer (1983). First Edition. 
	
	\bibitem{Bou} {\sc F. Bourseau}, {\it Die Isometrien der Oszillatorgruppe und einige ergebnisse \"uber Pr\"amorphismen liescher Algebren}. Fakult\"at f\"ur Mathematik der Universit\"at Bielefeld (1989).
	
	\bibitem{BOV} {\sc V. del Barco, \sc G. Ovando, \sc F. Vittone}, {\it Lorentzian compact manifolds: Isometries and geodesics}, J. Geom. Phys. {\bf 78}, 48--58 (2014).
	
	\bibitem{MF} {\sc M. Fischer}, {\it Lattices of oscillator groups}, J. Lie Theory {\bf 27} (1), 85--110 (2017). 	
		
	\bibitem{Ga} {\sc G.  Galloway}, {\it  Compact Lorentzian manifolds without closed nonspacelike geodesics}, Proc.
		Amer. Math. Soc. {\bf 98}, 119--123  (1986).
		
	\bibitem{Hel} {\sc S. Helgasson}, {\it Differential Geometry, Lie Groups, and Symmetric Spaces}, Graduate Studies in Mathematics, vol. {\bf 34}, American Math. Soc. (1999).
		
	\bibitem{Me} {\sc A. Medina}, {\it Groupes de Lie munis de m\'etriques bi-invariantes}. (Lie groups admitting bi-invariant metrics), T\^ohoku Math. J., II. Ser. {\bf 37}, 405--421 (1985). 
	
	\bibitem{MeRe} {\sc A. Medina,  P. Revoy}, {\it Les groupes oscillateurs et leurs r\'eseaux. (Oscillator groups and their lattices).}, Manuscr. Math. {\bf 52}, 81--95 (1985). 

	\bibitem{MU} {\sc D. M\"uller}, {\it Isometries of bi-invariant pseudo-Riemannian metrics on Lie groups},  Geom. Dedicata {\bf 29} (1),  65--96 (1989).

	\bibitem{ON} {\sc B. O'Neill}, {\it Semi-Riemannian geometry with
		applications to relativity}, Academic Press (1983).
	
	\bibitem{Ov} {\sc G. Ovando}, {\it Lie algebras with ad-invariant metrics- A survey}, In Memorian Sergio Console, Rendiconti del Seminario Matematico di Torino. {\bf  74}, 1-2, 241 -- 266 (2016).
	
	\bibitem{Ra} {\sc  M.S. Raghunathan}, Discrete Subgroups of Lie Groups, Springer Verlag, 1972.
	
	\bibitem{Su} {\sc S. Suhr}, {\it Closed geodesics in Lorentzian surfaces},  	Trans. Amer. Math. Soc. {\bf 365}, 1469--1486 (2013).
	
\end{thebibliography}

\appendix 

\end{document} 
